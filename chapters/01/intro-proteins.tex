\chapter{Introduction and proteins}

\section{Introduction}
Biomolecular modelling  has seen a recent increase in its use in the recent years, with a field still destined to expand.
Most of this models take a top-down approach, starting from the macroscopic rather than to build simulation from the fundamental and quantistic concepts.
Examples of systems studied through biomolecular modelling are:

\begin{multicols}{2}
	\begin{itemize}
		\item Channels.
		\item Photosynthetic systems.
		\item Viruses.
		\item DNA/RNA interactions.
		\item Inorganic systems.
	\end{itemize}
\end{multicols}

Through biomolecular modelling it is possible to obtain:

\begin{multicols}{2}
	\begin{itemize}
		\item Molecular rationale for biological processes like proteins' function or its misfolding.
		\item A quantitative evaluation of molecular driving forces.
		\item A prediction of properties of macromolecular structures and architectures.
		\item A comparative assessment of molecular affinities through the binding free energy.
	\end{itemize}
\end{multicols}

\section{Proteins}
Proteins have different functions within a cell:

\begin{multicols}{2}
	\begin{itemize}
		\item Give structure.
		\item Provide exchange of materials.
		\item Code for messages.
		\item Transport ions.
		\item Catalytic.
		\item Movement.
		\item Storage.
		\item Act as toxins.
	\end{itemize}
\end{multicols}

Proteins are a polymer of amino-acids and occupy a space-scale of $10nm$.
The amino-acids are in the range of $1nm$.
They are built through a polymerization reaction as chain of amino-acids coded through a degenerate code of RNA nucleotides.
Three bases of RNA code for an amino-acid.

	\subsection{Amino-acids}
	Amino-acids are the monomers of a protein.
	They have a general structure with an amino and a carboxyl terminal group for all of them.
	They are distinguished by a residue on the $\alpha$-carbon which gives them different chemical and physical properties.

	\subsection{Structure}
	There are four level of a protein structure.

	\begin{multicols}{2}
		\begin{itemize}
			\item Primary structure: the amino-acid sequence.
			\item Secondary structure: here $\alpha$-helices and $\beta$-sheet can be distinguished.
			\item Tertiary structure: the spatial, 3D dynamic configuration of a protein which arise during protein folding.
			\item Quaternary structure: the interaction of multiple correctly-folded proteins.
		\end{itemize}
	\end{multicols}
