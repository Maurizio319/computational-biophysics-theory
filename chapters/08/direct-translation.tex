\chapter{Direct translation}

\section{Introduction}
The direct translation method is a formal representation of the Verlet and the Velocity Verlet algorithms.
It has direct effect on those.

\section{Liouville operator}
The Liouville operator can be computed for any quantity $a$ that depends on the coordinates in phase space.
And its derivative is:

$$\frac{da}{dt} = \frac{\partial a}{\partial x_t}\dot{x}_t = \sum\limits_a\biggl[\frac{\partial a}{\partial q_a}\dot{q}_a + \frac{\partial a}{\partial p_a}\dot{p}_a\biggr] = \sum\limits_a\biggl[\frac{\partial a}{\partial q_a}\frac{\partial\mathcal{H}}{\partial p_a} - \frac{\partial a}{\partial p_a}\frac{\partial\mathcal{H}}{\partial q_a}\biggr] = \{a, \mathcal{H}\}$$

The Liouville operator is defined as:

$$iLa = \{a, \mathcal{H}\}\Rightarrow\frac{da}{dt} = iLa$$

The operation of the Liouville operator is to compute the Poisson brackets of $a$ with $\mathcal{H}$.
So that the dependence on time of $a$ can be written.
In abstract terms the operator acts on something and computes the Poisson brackets with the Hamiltonian: $iL\cdots = \{\cdots, \mathcal{H}\}$.
It can also be written as:

$$iL = \sum\limits_a\biggl[\frac{\partial\mathcal{H}}{\partial q_a}\frac{\partial}{\partial q_a} - \frac{\partial\mathcal{H}}{\partial q_a}\frac{\partial}{\partial p_a}\biggr]$$

A formal solution to the equation can be written:

$$\frac{da}{dt} = iLa\Rightarrow a(x_t) = e^{\overbrace{iLt}^{\text{Classical propagator}}}a(x_0)$$

Obtaining the value of $a(x_t)$.
It is called a classical propagator because it propagates $a$ from time $0$ to $t$.

	\subsection{Liouville operator split}
	This classical propagator is difficult to solve analytically so it needs to be approximated.
	The Liouville operator can be split in two operators:

	$$iL = \sum\limits_a\biggl[\frac{\partial\mathcal{H}}{\partial p_a}\frac{\partial}{\partial q_a} - \frac{\partial\mathcal{H}}{\partial q_a}\frac{\partial}{\partial p_a}\biggr] = iL_1 + iL_2$$

	$$iL_1 = \sum\limits_a\frac{\partial\mathcal{H}}{\partial p_a}\frac{\partial}{\partial q_a} \qquad iL_2 = -\sum\limits_a\frac{\partial\mathcal{H}}{\partial q_a}\frac{\partial}{\partial p_a}$$

	These two operators are non-commuting:

	$$iL_1iL_2\phi(x)\neq iL_2iL_1\phi(x)$$

	Formally, the commutator:

	$$iL_1iL_2-iL_2iL_1\equiv [iL_1, iL_2]$$

	If it is equal to zero the operators are commuting.
	In the case for the Liouville's operator split they are not commuting.

	\subsection{One dimensional example}
	To demonstrate that the two Liouville's operator are not commuting consider the Hamiltonian:

	$$\mathcal{H} = \frac{p^2}{2m} + U(x)$$

	So that the two operators are:

	$$iL_1 = \frac{\partial\mathcal{H}}{\partial p}\frac{\partial}{\partial x} = \frac{p}{m}\frac{\partial}{\partial x}\qquad iL_2 = -\frac{\partial\mathcal{H}}{\partial x}\frac{\partial}{\partial p} = -\frac{\partial U}{\partial x}\frac{\partial}{\partial p} = F(x)\frac{\partial}{\partial p}$$

	Applying the two operators in sequence to a function $\phi(x, p)$.
	Here the operator $iL_2$ is applied first:

	$$iL_1iL_2\phi(x, p) = \frac{p}{m}\frac{\partial}{\partial x}\biggl[F(x)\frac{\partial}{\partial p}\biggr]\phi(x, p) = \frac{p}{m}F(x)\frac{\partial^2\phi}{\partial x\partial p} + \frac{p}{m}F'(x)\frac{\partial\phi}{\partial p}$$

	Here the operator $iL_1$ is applied first:

	$$iL_2iL_1\phi(x, p) = F(x)\frac{\partial}{\partial p}\biggl[\frac{p}{m}\frac{\partial}{\partial x}\biggr]\phi(x, p) = \frac{p}{m}F(x)\frac{\partial^2\phi}{\partial x\partial p} + \frac{F(x)}{m}\frac{\partial \phi}{\partial x}$$

	Computing the commutator:

	$$[iL_1, iL_2]\phi(x, p) = \frac{p}{m}F'(x)\frac{\partial \phi}{\partial p} -\frac{F(x)}{m}\frac{\partial \phi}{\partial x}\neq 0$$

\section{Trotter theorem}
Since the two operators do not commute this equality is false:

$$iL_1iL_2\phi(x)\neq iL_2iL_1\phi(x)\Rightarrow e^{iLt}\neq e^{iL_1t}e^{iL_2t}$$

This is important because although the classical propagator cannot be written, in many situation the propagator corresponding to $iL_1$ and $iL_2$ can be derived, so that the classical propagator could be obtained.
However, according to the Trotter theorem:

$$e^{A+B} = \lim\limits_{P\rightarrow\infty}[e^{\frac{B}{2P}}e^{\frac{A}{P}}e^{\frac{B}{2P}}]^P$$

Therefore, considering the Liouville's split:

$$e^{iLt} = e^{iL_1t+iL_2t} = \lim\limits_{P\rightarrow\infty}[e^{\frac{iL_2t}{2P}}e^{\frac{iL_1t}{P}}e^{\frac{iL_2t}{2P}}]^P$$

	\subsection{$\Delta t = \frac{t}{P}$}

	$$e^{iLt} = e^{iL_1t+iL_2t} = \lim\limits_{P\rightarrow\infty}[e^{\frac{iL_2t}{2P}}e^{\frac{iL_1t}{P}}e^{\frac{iL_2t}{2P}}]^P$$

	This formula  can be written as:

	$$e^{iLt} = e^{iL_1t+iL_2t} = \lim\limits_{P\rightarrow\infty, \Delta t\rightarrow 0}[e^{\frac{iL_2\Delta t}{2}}e^{iL_1\Delta t}e^{\frac{iL_2\Delta t}{2}}]^P$$

	Assuming that $\Delta t = \frac{t}{P}$.
	When a simulation there will be a finite value of $P$, which will be a natural value:

	$$e^{iLt}\approx[e^{\frac{iL_2\Delta t}{2}}e^{iL_1\Delta t}e^{\frac{iL_2\Delta t}{2}}]^P + \underbrace{o(P\Delta t^3)}_{\text{Global error}}$$

	So that the classical propagator is equal to the term in the limit with an error due to the truncation to a finite value of $P$.
	Taking the $\frac{1}{P}$ power:

	$$e^{iLt}\approx e^{\frac{iL_2\Delta t}{2}}e^{iL_1\Delta t}e^{\frac{iL_2\Delta t}{2}} + \underbrace{o(\Delta t^3)}_{\text{Local error}}$$

	For a small time step $\Delta t$.
	So following a trajectory for a huge number of step the error is $o(\Delta t^2)$.
	In this way an algorithm that comes from Trotter factorization can be built.
	This algorithm will have a limited global error of the order of $o(\Delta t^2)$, so it will have a strong numerical stability.

	\subsection{One-dimensional example}
	In one-dimensional example everything can be written as:

	$$iL_1 = \frac{\partial\mathcal{H}}{\partial p}\frac{\partial }{\partial x} = \frac{p}{m}\frac{\partial }{\partial x}\qquad iL_2 = F(X)\frac{\partial}{\partial p}$$

	So the Trotter algorithm can be written as:

	$$e^{iLt}\approx e^{\frac{iL_2\Delta t}{2}}e^{iL_1\Delta t}e^{\frac{iL_2\Delta t}{2}} + o(\Delta t^3)$$

	And making everything explicit:

	$$e^{iL\Delta t}\approx e^{\frac{\Delta t}{2}F(x)\frac{\partial}{\partial p}}e^{\Delta t\frac{p}{m}\frac{\partial}{\partial x}}e^{\frac{\Delta t}{2}F(x)\frac{\partial}{\partial p}} + o(\Delta t^3)$$

	Applying the operator to the coordinates in phase space:

	$$\begin{pmatrix}x(\Delta t)\\ p(\Delta t)\end{pmatrix}=e^{iL\Delta t}\begin{pmatrix}x(\Delta t)\\ p(\Delta t)\end{pmatrix}\approx e^{\frac{\Delta t}{2}F(x)\frac{\partial}{\partial p}}e^{\Delta t\frac{p}{m}\frac{\partial}{\partial x}}e^{\frac{\Delta t}{2}F(x)\frac{\partial}{\partial p}}\begin{pmatrix}x(\Delta t)\\ p(\Delta t)\end{pmatrix}$$

	\subsection{Exponential operators}
	An exponential operator can be computed as:

	$$e^{c\frac{\partial}{\partial x}}g(x) = \sum\limits_{k=0}^\infty\frac{1}{k!}\biggl(c\frac{\partial}{\partial x}\biggr)^kg(x) = \sum\limits_{k=0}^\infty\frac{1}{k!}c^k\frac{d^kg(x)}{dx^k} = g(x+c)$$

	This is another way to compute the Taylor expansion of $g$ in $x+c$.
	So the operator translates the argument of $g$ by a quantity $c$.
	Now applying the exponential operators to the points:

	$$\begin{pmatrix}x(\Delta t)\\p(\Delta t)\end{pmatrix} = e^{\frac{\Delta t}{2}F(x(0))\frac{\partial}{\partial p(0)}}e^{\Delta t\frac{p(0)}{m}\frac{\partial}{\partial x(0)}}e^{\frac{\Delta t}{2}F(x(0))\frac{\partial}{\partial p(0)}}\begin{pmatrix}x(0)\\p(0)\end{pmatrix}$$

	Computing the first exponential operator:

	$$e^{\frac{\Delta t}{2}F(x(0))\frac{\partial}{\partial p(0)}}\begin{pmatrix}x(0)\\p(0)\end{pmatrix} = \begin{pmatrix}x(0)\\p(0) + \frac{\Delta t}{2}F(x(0))\end{pmatrix}$$

	Now doing the same for all the other operators:

	\begin{align*}
		\begin{pmatrix} x(\Delta t)\\ p(\Delta t)\end{pmatrix} &= e^{\frac{\Delta t}{2}F(x(0))\frac{\partial}{\partial p(0)}}e^{\Delta t\frac{p(0)}{m}\frac{\partial }{\partial x(0)}}\begin{pmatrix} x(0)\\p(0) + \frac{\Delta t}{2}F(x(0))\end{pmatrix} =\\
																													 & = e^{\frac{\Delta t}{2}F(x(0))\frac{\partial}{\partial p(0)}}\begin{pmatrix}x(0) + \Delta t\frac{p(0)}{m}\\p(0) + \frac{\Delta t}{2}F(x(0) + \Delta t\frac{p(0)}{m})\end{pmatrix} =\\
																													 &= \begin{pmatrix} x(0) + \frac{\Delta t}{m}\biggl(p(0) + \frac{\Delta t}{2}F(x(0))\biggr) \\ p(0) + \frac{\Delta t}{2}F(x(0)) + \frac{\Delta t}{m}\biggl(x(0) + \frac{\Delta t}{m}\biggl(p(0) + \frac{\Delta t}{2}F(x(0))\biggr)\biggr)\end{pmatrix}
	\end{align*}

\section{Trotter algorithm}
This is the result of Trotter algorithm:

$$\begin{pmatrix} x(\Delta t)\\ p(\Delta t)\end{pmatrix} =\begin{pmatrix} x(0) + \frac{\Delta t}{m}\biggl(p(0) + \frac{\Delta t}{2}F(x(0))\biggr) \\ p(0) + \frac{\Delta t}{2}F(x(0)) + \frac{\Delta t}{m}\biggl(x(0) + \frac{\Delta t}{m}\biggl(p(0) + \frac{\Delta t}{2}F(x(0))\biggr)\biggr)\end{pmatrix}$$

Looking at the first line:

$$x(\Delta t) = x(0) + \frac{\Delta t}{m}\biggl(p(0) + \frac{\Delta t}{2}F(x(0))\biggr)\Rightarrow x(\Delta t) = x(0) + v(0)\Delta t + \frac{\Delta t^2}{2m}F(0)$$

And considering the second line:

$$p(\Delta t) = p(0) + \frac{\Delta t}{2}F(x(0)) + \frac{\Delta t}{2}F\biggl(x(0) + \frac{\Delta t}{m}\biggl(p(o) + \frac{\Delta t}{2}F(x(0))\biggr)\biggr)$$

So that:

$$p(\Delta t) = p(0) + \frac{\Delta t}{2}[F(x(0)) + F(x(\Delta t))]\Rightarrow v(\Delta t) = v(0) + \frac{\Delta t}{2m}[F(0) + F(\Delta t)]$$

So that inside the square brackets it is an average of the forces at time $0$ and at time $\Delta t$.
The two final formulae are the one needed to update the positions and velocities of the system.
These two formulas are the Velocity Verlet algorithm.
This is an important result telling that the Velocity Verlet is the best algorithm for molecular dynamics.

	\subsection{Direct translation}
	There is a quick way to compute the results of the Trotter algorithm, which leads to direct translation.
	What there is to do is to work in a sloppy matter on the operators.
	In this way the formulae can be written directly into lines of code.

		\subsubsection{Momentum translation}
		Considering the first operator, it is translating the momentum by the quantity $\frac{\Delta t}{2}F(x)$.
		So this will translate $p$ from time zero to time $\frac{\Delta t}{2}$.
		So when applying this operator $p$ will change and $x$ will remain the same.

		$$e^{\frac{\Delta t}{2}F(x)\frac{\partial}{\partial p}}\begin{pmatrix}x(0)\\p(0)\end{pmatrix} = \begin{pmatrix} x(0)\\ p\biggl(\frac{\Delta t}{2}\biggr) = p(0) + \frac{\Delta t}{2}F(x(0))\end{pmatrix}$$

		Corresponding code:

		\begin{lstlisting}[escapeinside={(*}{*)}]
			p = p + 0.5 * (*$\Delta t$*) * F
		\end{lstlisting}

		\subsubsection{Position translation}
		Applying the second operator only $x$ will be translated:

		$$e^{\frac{\Delta}{m}p\biggl(\frac{\Delta t}{2}\biggr)\frac{\partial}{\partial x}}\begin{pmatrix}x(0)\\p\biggl(\frac{\Delta t}{2}\biggr) = p(0) + \frac{\Delta t}{2}F(x(0))\end{pmatrix}$$

		Corresponding code:

		\begin{lstlisting}[escapeinside={(*}{*)}]
			x = x + (*$\Delta t$*) * F * p/m
		\end{lstlisting}

		\subsubsection{Momentum translation with updated forces}
		Now applying the third operator $p$ will be translated by another $\frac{\Delta t}{2}$.
		Looking at this formula the same result of the Trotter theorem is obtained.
		In this third step also the forces are updated.

		Corresponding code:

		$$e^{\frac{\Delta t}{2}F(x)\frac{\partial}{\partial p}}\begin{pmatrix}x(\Delta t)\biggl(\frac{\Delta t}{2}\biggr)\end{pmatrix} = \begin{pmatrix} x(\Delta t) \\ p(\Delta t) = p\biggl(\frac{\Delta t}{2}\biggr) + \frac{\Delta t}{2}F(x(\Delta t))\end{pmatrix}$$

		\begin{lstlisting}[escapeinside={(*}{*)}]
			p = p + 0.5 * (*$\Delta t$*) * F
		\end{lstlisting}

	\subsection{Multiple time step integration}
	The problem when studying a force field, there are several term that correspond to the bonded interactions and the non-bonded interactions:

	\begin{align*}
		U(\vec{r}_1, \dots, \vec{r}_N) =& \frac{1}{2}\sum\limits_{bonds} K_{bond}(r - r_0)^2 + \frac{1}{2}\sum\limits_{bends}K_{bend}(\theta - \theta_0)^2 + \sum\limits_{torsions}\sum\limits_{n=0}^6 A_n[1 + \cos(C_n\phi + \delta_n)] + \\
																		&+\sum\limits_{i, j\in nb}\biggl\{\biggl[ 4\epsilon_{ij}\biggl(\frac{\sigma_{ij}}{r_{ij}}\biggr)^{12}-\biggl(\frac{\sigma_{ij}}{r_{ij}}\biggr)^6\biggr] + \frac{q_iq_j}{r_{ij}}\biggr\}
	\end{align*}

	It can be assumed that the atoms will be subject to forces with different time scales: the bonded interactions will change more rapidly than the non-bonded interactions.
	This is because the positions on the atoms do not change so much.
	It can be seen how the non-bonded interaction, which are the slow forces are very expensive to compute.
	So there is no reason to recompute the non-bonded interactions every time the fast forces are changing.
	So the Hamiltonian can be split into a reference Hamiltonian corresponding to the fast forces and another corresponding to the slow ones:

	$$\mathcal{H}_{ref} = \frac{p^2}{2m} + U_{fast}(x)\qquad \dot{x} = \frac{p}{m}\qquad \dot{p} = F_{fast}(x) + F_{slow}(x)$$

	In this case the Liouville's operator will be:

	$$iL = \frac{p}{m}\frac{\partial }{\partial x} + [F_{fast}(x) + F_{slow}(x)]\frac{\partial}{\partial p}$$

	So that there will be a fast Liouville's operator and a slow one.

	\subsection{Reference system propagator}
	The reference system propagator is referred to the reference Hamiltonian.
	Considering:

	$$iL = \frac{p}{m}\frac{\partial }{\partial x} + [F_{fast}(x) + F_{slow}(x)]\frac{\partial}{\partial p} = iL_{fast} + iL_{slow}$$

	And:

	$$\mathcal{H}_{ref} = \frac{p^2}{2m} + U_{fast}(x)$$

	The reference system propagation for the fast forces and the slow part of the slow propagator:

	$$iL_{fast} = \frac{p}{m}\frac{\partial}{\partial x} + F_{fast}(x)\frac{\partial}{\partial p}\qquad iL_{slow} = F_{slow}(x)\frac{\partial}{\partial p}$$

	Applying Trotter factorisation:

	$$e^{iL\Delta t} = e^{iL_{slow}\frac{\Delta t}{2}}e^{iL_{fast}\Delta t}e^{iL_{slow}\frac{\Delta t}{2}}$$

	\subsection{RESPA}
	RESPA is the reference system propagator algorithm, where $\Delta t$ is chosen according to the time scale of the slow forces, allowing to use a much bigger $\Delta t$ with respect to the one considering all the forces together.
	It comes from the split of the Hamiltonian into a slow and fast part.
	It allow to compute the electrostatic and Van der Waals forces less times than the calculation of the bonded interactions, saving a lot of computational power.
	Focussing on the fast part of the RESPA algorithm:

	$$e^{iL_{fast}\Delta t} = \biggl[e^{\frac{\delta t}{2}F_{fast}\frac{\partial}{\partial p}}e^{\delta t\frac{p}{m}\frac{\partial}{\partial x}}e^{\frac{\delta t}{2}F_{fast}\frac{\partial}{\partial p}}\biggr]^n\qquad \delta t = \frac{\Delta t}{n}$$

	This part can be factorized again using the fact that $\Delta t$ is sub-divided in $n$ intervals, where $\delta t$ is the time-step where the fast forces will act.
	This will be applied $n$ times so it corresponds to one time step.
	Writing down the entire classical propagator the fast operator is in-between the slow one.

	$$e^{iL\Delta t} = e^{\frac{\Delta t}{2}F_{slow}\frac{\partial}{\partial p}}\biggl[e^{\frac{\delta t}{2}F_{fast}\frac{\partial}{\partial p}}e^{\delta t\frac{p}{m}\frac{\partial}{\partial x}}e^{\frac{\delta t}{2}F_{fast}\frac{\partial}{\partial p}}\biggr]^ne^{\frac{\Delta t}{2}F_{slow}\frac{\partial}{\partial p}}$$

	And for a direct translation in pseudocode:

	\begin{lstlisting}[escapeinside={(*}{*)}]
		p = p + 0.5 * (*$\Delta t$*) * (*$F_{slow}$*)
		for i = 1 to n
			p = p + 0.5 * *($\delta t$*) * (*$F_{fast}$*)
			x = x + (*$\delta t$*) * p/m
			Recompute only fast forces
			p = p + 0.5 * (*$\delta t$*) * (*$F_{fast}$*)
		endfor
		Recompute slow forces
		p = p + 0.5 * (*$\delta t$*) * (*$F_{slow}$*)
	\end{lstlisting}

\section{Symplectic solver}
By demonstrating the applicability of the Trotter algorithm, the symplectic property of the Verlet algorithm has been demonstrated.
The Hamiltonian is not strictly conserved, however it conserves a shadow Hamiltonian $\tilde{\mathcal{H}}(x, \Delta t)$ that is close to the true Hamiltonian $\mathcal{H}(x)$ such that:

$$\lim\limits_{\Delta t\rightarrow 0}\tilde{\mathcal{H}}(x, \Delta t) = \mathcal{H}(x)$$

This property guarantees that this algorithm is numerically stable for any amount of time.
