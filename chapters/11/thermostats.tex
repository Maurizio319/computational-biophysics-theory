\chapter{Thermostats}

\section{Introduction}
When dealing with computer simulation of a biomolecule temperature has to be constant, so the effect of a thermostat has to be included.
So the particles will interact with the particles in the thermal bath.
The idea is to check whether the kinetic energy is what is expected for the particular temperature and take action when that is not the case.

	\subsection{Velocity rescaling}
	The velocity rescaling technique involve having a target kinetic energy compatible with the temperature.
	Considering $N_f$ the number of degrees of freedom.

	$$\bar{K} = \frac{N_f}{2\beta}$$

	This is what is expected from the equipartition theorem.
	When running a simulation, the actual kinetic energy is:

	$$K = \frac{1}{2}\sum\limits_i m_i\vec{v}_i^2$$

	The objective is to have $K = \bar{K}$, so the velocities are rescaled by a rescaling factor$\alpha = \sqrt{\frac{\bar{K}}{K}}$ where $\vec{v}_i \rightarrow \frac{\vec{v}_i}{\alpha}$.
	This allows to obtain the rescaled kinetic energy:

	$$\frac{1}{2}\sum\limits_{i}m_i\frac{\vec{v}_i^2}{\alpha^2} = \frac{\bar{K}}{2K}\sum\limits_{i}m_i\vec{v}_i^2 = \bar{K}$$

	So that the target kinetic energy is obtained in the simulation.
	Velocities are rescaled at a predetermined frequency or using a threshold value on the actual kinetic energy to allow for its fluctuations.
	This algorithm either provide more velocities to particles or it slows them down.
	In the first case there will be problems in satisfying all the constraints.
	This kind of thermostat is rigorously correct only in the thermodynamics limit: $N\rightarrow\infty$.
	It is not recommended for a small number of particles.

\section{Type of thermostats}

	\subsection{Andersen's thermostat}
	Consider a collision frequency $\nu$, then $\nu\Delta t$ is the probability of a collision in $\Delta t$.
	In particular every $\frac{1}{\nu}$ there will be a collision.
	To implement the changes in kinetic energy considering the probability $\nu\Delta t$, a random number $\rho$ from a uniform probability distribution in $[0,1]$, if $\rho<\nu\Delta t$ the velocity is reassigned to the particle, extracting it from the Maxwell-Boltzmann distribution:

	$$P(p) = \biggl(\frac{\beta}{2\pi m}\biggr)^{\frac{3}{2}}e^{-\beta\frac{p^2}{2m}}$$

	So this distribution samples the canonical ensemble.
	This is done for all the particles.
	This causes abrupt discontinuity in the velocities, making it difficult to satisfy the constraints.

	\subsection{Andersen revisited - Heyes}
	A modification on Andersen thermostat has been done by Heyes.
	There is a mixture between the velocity algorithm and Andersen's thermostat.
	The target kinetic every $K_t$ is selected from a distribution:

	$$P(K_t)dK_t\propto K_t^{\frac{N_f}{2}-1}e^{-\beta K_t}d K_t$$

	This is what is would be obtained from the Maxwell-Boltzmann distribution, corrected by the number of degrees of freedom.
	The target kinetic energy is not fixed, but it will vary and then the velocities are re-scaled.
	It is more gentle than the Andersen's thermostat and it is more rigorous than the velocities rescaling because the target kinetic energy can vary according to the distribution.
	Once the actual kinetic energy is obtained:

	$$K = \frac{1}{2}\sum\limits_im_i\vec{v}_i^2$$

	A scaling factor is computed $\alpha = \sqrt{\frac{K_t}{K}}$ and the velocities are updated: $\vec{v}_i \rightarrow \frac{\vec{v}_i}{\alpha}$ at a predetermined frequency or considering some threshold.
	The rescaled kinetic energy will then be:

	$$\frac{1}{2}\sum\limits_{i}m_i\frac{\vec{v}_i^2}{\alpha^2} = \frac{K_t}{2K}\sum\limits_{i}m_i\vec{v}_i^2 = K_t$$

	This is rigorously correct in the thermodynamics limit: $N\rightarrow\infty$
	The thermodynamics limit will be reached sooner than the velocity rescaling algorithm.

	\subsection{Langevin thermostat}
	The Langevin thermostat is based on Langevin equation.
	Friction with a friction coefficient $\gamma$ is introduced, dissipating some of the energy and also some random velocities $\eta_i$ are introduced to keep the temperature and the kinetic energy fixed.
	The random velocities model the collisions between system and thermal bath particles.
	The Langevin equations will be used for both the solute and solvent particles.
	Using Langevin's thermostat the canonical solution is sampled.
	The properties of the random velocities are:

	\begin{multicols}{2}
		\begin{enumerate}
			\item Collisions between solute and solvent and thermal bath particles are many and independent.
				Components of the random velocities are sampled from a Gaussian distribution.
			\item On average there is no net momentum exchange between the thermal bath and the solute particles: $\langle\eta_i(t)\rangle = 0$.
			\item Frequent and instantaneous collisions.
				To have instantaneous collisions the auto correlation function of the velocities decays fast: $\langle\eta_i(t)\eta_j(t)\rangle = 6D\delta_{ij}\delta(t-t')$ where $D = \frac{kT}{m\gamma}$ is the diffusion coefficient.
		\end{enumerate}
	\end{multicols}

	Now everything is implemented considering the equation of motions.
	The equations of motion are coupled overdamped Langevin equations.
	There is no acceleration because in the overdamped limit the mass term is negligible and the change in velocities is:

	$$\dot{\vec{r}}_i(t) = \beta D\vec{F}_i(\vec{r}_1(t), \dots, \vec{r}_N(t)) + \eta_i(t)$$

	This is used for computing the velocities in the Verlet algorithm.
	Writing this down the configuration obtained will sample the Fokker-Planck equation:

	$$\frac{\partial}{\partial t}P(\vec{x}, t) = D\sum\limits_{i=1}^{N}\nabla_i\cdot[\nabla_i+\beta\nabla\mathcal{H}(\vec{x})]$$

	The Boltzmann distribution $P(\vec{x}) = Ce^{-\beta\mathcal{H}(\vec{x})}$ is the stationary solution of the Fokker-Planck equation.
	Guaranteeing that in the long time limit the solution the velocities will be sampled from the Boltzmann distribution.
	This implies that simulations have to be long and the effect of the random velocities has to be considered.
	To integrate the Langevin equation:

	$$\dot{\vec{r}}_i(t) = \beta D\vec{F}_i(\vec{r}_1(t), \dots, \vec{r}_N(t)) + \eta_i(t)$$

	$$\langle\eta_i(t)\eta_j(t)\rangle = 6D\delta_{ij}\frac{\delta_{kl}}{\Delta t}$$

	The random velocity is sampled from a Gaussian distribution with a non-unitary variance:

	$$\sigma^2 = \langle\eta_i(0)\eta_i(0)\rangle = \frac{6D}{\Delta t}$$

	Where $6$ is $2$ times the number of dimensions.
	Now a random force $\xi_i(t_k)$ satisfying the unitary variance condition:

	$$\xi_i(t_k) = \sqrt{\frac{\Delta t}{6D}}\eta_i(t_k)\qquad\langle\xi_i(t_k)\rangle = 0\qquad\langle\xi_i(t_k)\xi_i(t_l)\rangle = \delta_{ij}\delta_{kl}$$

	Random velocities are rescaled to obtain a variance equal to $1$.
	Considering a discretized Langevin equation:

	$$\frac{\vec{r}_i(t+ \Delta t) - \vec{r}_i(t)}{\Delta t} = \beta D\vec{F}_i(\vec{r}_1(t), \dots, \vec{r}_N(t)) + \sqrt{\frac{6D}{\Delta t}}\xi_i(t)$$

	Thus:

	$$\vec{r}_i(t + \Delta t) = \vec{r}_i(t) + \Delta t\frac{\vec{F}_i(\vec{r}_1(t), \dots, \vec{r}_N(t))}{m_i\gamma} + \sqrt{\frac{6kT\Delta t}{m_i\gamma}}\xi_i(t)$$

	$\xi_i$ are random numbers with average $0$ and variance $1$, allowing to obtain the coordinate at time $t+\Delta t$, getting a stationary solution in the long term of the Fokker-Planck equation, which is the Boltzmann distribution.

\section{Bussi velocity Verlet}
The Bussi velocity Verlet is a generalized version of the velocities rescaling based on velocity rescale with the following algorithm:

\begin{multicols}{2}
	\begin{itemize}
		\item Evolve the system for a single time step with Hamilton's equations, using a time-reversible area-preserving integration scheme like the velocity Verlet or any integration scheme for the microcanonical ensemble.
		\item Compute the kinetic energy, which becomes a variables with its own dynamics.
		\item Evolve the kinetic energy for a time corresponding to a single time step using an auxiliary continuous stochastic dynamics equations.
		\item Rescale the velocities so as to enforce this new value of the kinetic energy.
	\end{itemize}
\end{multicols}

The equations that represent the auxiliary dynamics on kinetic energy:

$$dK = \biggl(D(K)\frac{\partial\log P(K)}{\partial K}+\frac{\partial D(K)}{\partial K}\biggr)dt + \sqrt{2D(K)}dW$$

Where $dW$ is a stochastic variation and the first part is completely deterministic.
The probability distribution for the kinetic energy is taken from Heyes' algorithm: $P(K_t)dK_t\propto K_t^{\frac{N_f}{2}-1}e^{\beta K_t}dK_t$:

$$dK = \biggl(\frac{N_f D(K)}{2K\bar{K}}(K-\bar{K}) - \frac{D(K)}{K} + \frac{\partial D(K)}{\partial K}\biggr) dt + \sqrt{2D(K)}dW$$

A possible choice for $D(K)$ is:

$$D(K) = \frac{2K\bar{K}}{N_f\tau}$$

So that:

$$dK = (K-\bar{K})\frac{dt}{\tau}+\sqrt{\frac{2K\bar{K}}{N_f}}\frac{dW}{\sqrt{\tau}}$$

This is the auxiliary variable stochastic dynamics equation for the kinetic energy.
Neglecting the stochastic term Berendsen's thermostat is obtained:

$$dK = (K-\bar{K})\frac{dt}{\tau}$$

Here the kinetic energy changes only when it is different from $\bar{K}$.
This is not sampling the canonical distribution.
So there is no known ensemble that can be simulated.
This is useful for simulating at a given temperature and to perform an equilibration step.
Once equilibrium is obtained a proper thermostat is used.
Reconsidering the stochastic part there are two important limits:

$$\tau\rightarrow 0\Rightarrow \tau dK = (K-\bar{K})dt + \sqrt{\frac{2K\bar{K}\tau}{N_f}}dW\Rightarrow (K-\bar{K})dt = 0 \text{ (Heyes)}$$

Sampling the kinetic energy from the Heyes distribution.
The other limit is when:

$$\tau\rightarrow\infty\Rightarrow dK = (K-\bar{K})\frac{dt}{\tau} + \sqrt{\frac{2K\bar{K}}{N_f}}\frac{dW}{\sqrt{\tau}}\Rightarrow dK = 0\text{ (Hamiltonian dynamics)}$$

In this way Hamiltonian dynamics are obtained: the kinetic energy will be treated as any other observable and has no dynamics.
If $\tau$ is in between the two values a family of algorithms with all possible $\tau$ are obtained.
The solution to the differential equation leads to a scaling factor ($R_i$ independent Gaussian random numbers):

$$\alpha^2 = e^{-\frac{\Delta t}{\tau}} + \frac{\bar{K}}{N_fK}(1-e^{-\frac{\Delta t}{\tau}})\sum\limits_{i=1}^{N_f}R_i^2+2e^{-\frac{\Delta t}{2\tau}}\sqrt{\frac{\bar{K}}{N_f K}(1-e^{-\frac{\Delta t}{\tau}}R_1)}$$

The scaling factor on the velocities depends on a sequence of random numbers, one for each degree of freedom.
This has the same computational cost as the velocity rescaling algorithm with two advantages:

\begin{multicols}{2}
	\begin{itemize}
		\item The canonical distribution is obtained.
		\item The velocities remain continuous and there are no problem with the constraints.
	\end{itemize}
\end{multicols}

\section{Nos\`e Hamiltonian}
Nos\`e Hamiltonian or the extended phase space involves to introducing an extra variable introducing a Maxwell daemon and when it sees that the particles have a higher velocity than expected it will assign a new velocity according to the Maxwell distribution keeping the temperature constant.
Doing so an extra variable in the Hamiltonian $s$ is obtained, for which its conjugated momentum $p_s$ is obtained.

$$\mathcal{H}_N = \sum\limits_i\frac{\vec{p}_i^2}{2m_is^2} + U(\vec{r}_1, \dots, \vec{r}_n)_ \frac{p_s^2}{2Q}+ gkT\log s$$

Where $g$ is a parameter.
For the conjugate momentum there is an extra term and $Q$ works like mass and is an inertial term for $p_s$, the bigger $Q$ the smaller the variation in kinetic variation.
In total there are $6N + 2$ variables in the system.
Assuming that this is the Hamiltonian, following Hamilton's equation the microcanonical ensemble is being simulated.
Considering the microcanonical partition function for Nos\`e Hamiltonian:

$$\Omega = \int d^N\vec{r}d^N\vec{p}dsdp_s\delta\biggl(\sum\limits_{i}\frac{\vec{p}_i^2}{2m_is^2} + U(\vec{r}_1, \dots, \vec{r}_N) + \frac{p_s^2}{2Q} + gkT\log s - E\biggr)$$

Where $\Omega$ is the total number of configuration.
The integration is over all the coordinates and the momenta and also on the extra coordinate and its conjugate momentum that depends on Maxwell's daemon.
The delta function is the function of the Nos\`e Hamiltonian minus the energy.
Scaling the momentum $\frac{\vec{p}}{s}\rightarrow \vec{p}$:

$$\Omega = \int d^N\vec{r}d^N\vec{p}dsdp_ss^{dN}\delta\biggl(\mathcal{H}+\frac{p_s^2}{2Q} + gkT\log s - E\biggr)$$

Where $N$ is the dimensionality of the system.
This can be solved using the property of the $\delta$ function, allowing to integrate over the two new variables:

$$\delta(f(s)) = \frac{\delta(s-s_0)}{|f'(s_0)|}\qquad f(s) = \mathcal{H}+\frac{p_s^2}{2Q} + gkT\log s - E\qquad f'(s) = \frac{gkT}{s}$$

Computing the zero of $f(s)$:

$$s_0 = e^{\frac{1}{gkT}\bigl(E-\mathcal{H}-\frac{p_s^2}{2Q}\bigr)}\Rightarrow f'(s_0) = gkTe^{-\frac{1}{gkT}\bigl(E-\mathcal{H}-\frac{p_s^2}{2Q}\bigr)}$$

Substituting $\delta(f(s))$ with $\frac{\delta(s-s_0)}{|f'(s_0)|}$:

$$\delta(f(s)) = \frac{1}{gkT}e^{\frac{1}{gkT}\bigl(E-\mathcal{H}-\frac{p_s^2}{2Q}\bigr)}\delta\biggl(s-e^{\frac{1}{gkT}\bigl(E-\mathcal{H}-\frac{p_s^2}{2Q}\bigr)}\biggr)$$

Plugging it in the microcanonical partition function:

$$\Omega = \int d^N\vec{r}d^N\vec{p}dsdp_ss^{dN}\delta\biggl(\mathcal{H}+\frac{p_2^2}{2Q}+gkT\log a-E\biggr)$$

Computing the integral in $s$ the net result is substituting $s$ with the value of $s_0$:

$$\Omega = \frac{1}{dkT}\int d^N\vec{r}d^N\vec{p}dp_se^{\frac{dN+1}{gkT}\bigl(E-\mathcal{H}-\frac{p_s^2}{2Q}\bigr)}$$

Determining the value of the parameter $g$ and considering the integral in $p_s$ this is a Gaussian integral:

$$g = dN + 1\Rightarrow\Omega = \frac{e^{\frac{E}{kT}}\sqrt{2\pi QkT}}{(dN+1)kT}\int d^N\vec{r}d^N\vec{p}e^{-\frac{\mathcal{H}}{kT}}$$

So that in the integral is remaining the partition function of the canonical ensemble.
Using the Nos\`e Hamiltonian the canonical ensemble in the original Hamiltonian is sampled.
The microcanonical partition function for the extended Nos\`e Hamiltonian corresponds to the canonical partition function for the original Hamiltonian non considering multiplicative factors.

	\subsection{Nos\`e equations}
	Considering the original Nos\`e Hamiltonian

	$$\mathcal{H}_N = \sum\limits_i\frac{\vec{p}_i^2}{2m_is^2}+U(\vec{r}_1,\dots, \vec{r}_N) + \frac{p_s^2}{2Q} + gkT\log s$$

	Applying Hamilton's equation to the coordinates and the momenta:

	$$\dot{\vec{r}}_i = \frac{\partial\mathcal{H}_N}{\partial\vec{p}_i} = \frac{\vec{p}_i}m_is^2\qquad \dot{\vec{p}}_i = -\frac{\partial\mathcal{H}_n}{\partial\vec{r}_i} = -\frac{\partial U}{\partial\vec{r}_i} = \vec{F}_i$$

	There are two extra equations that corresponds to the time evolution of the extra coordinates and of its conjugate momentum:

	$$\dot{s} = \frac{\partial\mathcal{H}_N}{\partial p_s} = \frac{p_s}{Q}\qquad \dot{p}_s = -\frac{\partial\mathcal{H}_n}{\partial S} = \sum\limits_i\frac{\vec{p}_i^2}{m_is^3} - \frac{gkT}{s} = \frac{1}{s}\biggl[\sum\limits_i\frac{\vec{p}_i^2}{m_is^2}-gkT\biggr]$$

	\subsection{Nos\`e-Hoover equations}
	Performing a noncanonical change of variables, so that there is no way that the new equations can be obtained starting from the Hamiltonian.

	$$\vec{p}_i' = \frac{\vec{p}_i}{s}\qquad p_s' = \frac{p_s}{s}\qquad dt' = \frac{dt}{s}$$

	Changing the derivative of position with the new time and momentum:

	$$\frac{d\vec{r}_i}{dt} = \frac{\vec{p}_i}{m_is^2}\Rightarrow\frac{d\vec{r}_i}{dt'} = \frac{\vec{p}_i'}{m_i}\qquad \frac{d\vec{p}_i'}{dt'} = \vec{F}_i-\frac{sp'_s}{Q}\vec{p}_i'$$

	The derivative of positions become the one in the original Hamiltonian.
	The derivative of momenta has the original force and a correction on it, extra forces proportional to the velocities and that look like friction terms, but it is not guaranteed to be a negative contribution.
	This can allow the temperature to be constant.

	$$\frac{ds}{dt'} = \frac{s^2p_s'}{Q}$$

	$$\frac{dp_s'}{dt'} = \frac{1}{s}\biggl[\sum\limits_{i} \frac{(\vec{p}_i')^2}{m_i}-gkT\biggr] - \frac{s(p_s')^2}{Q}$$

	These are Nos\`e equations and are computed applying Hamilton's equation to it.
	Now in this new form the look similar to Hamilton's equation to the original Hamiltonian.
	Performing another change of variables:

	$$\frac{1}{s}\frac{ds}{dt'} = \frac{d\eta}{dt'}\qquad p_s = p_\eta = sp'_s$$

	And starting from the equations a new set is obtained:

	\begin{multicols}{2}
		\begin{itemize}
			\item $\dot{\vec{r}}_i = \frac{\vec{p}_i}{m_i}$.
			\item $\dot{\vec{p}}_i = \vec{F}_i-\frac{p_\eta}{Q}\vec{p}_i$.
			\item $\dot{\eta} = \frac{p_\eta}{Q}$.
			\item $\dot{p}_\eta = \sum\limits_{i}\frac{\vec{p}_i^2}{m_i}-dNkT$.
				This is exactly twice the kinetic energy and the difference between this term and the double of the kinetic energy.
				Whenever the kinetic energy differs from the expected then the momentum $\dot{p}_\eta$ and that will act as a friction term.
		\end{itemize}
	\end{multicols}

These are the dynamics of the thermostat.

\section{Non Hamiltonian statistical mechanics}
The Nos\`e-Hoover cannot be derived from an Hamiltonian.
The equations were obtained from a non-canonical transformation of variables.
There is a need to define what happens in non-Hamiltonian systems.
Liouville's equations can be extended to non-Hamiltonian systems.
When describing the system and its surroundings the universe can be thought as an isolated system, which can be described through Hamiltonian dynamics: the energy is fixed.
When describe the system without the surrounding all the variables of the second have to be integrated out so they do not appear in the results.
This operation lead to obtain non-Hamiltonian dynamics.
The non-Hamiltonian equations of motion are:

$$\dot{x} = \xi(x, t)$$

This is the velocity in phase space.
Considering phase space compressibility for non-Hamiltonian dynamics:

$$\nabla\cdot\dot{x} = \nabla\cdot\xi(x, t) = \kappa(x, t)\neq 0$$

Where $\kappa$ is the compressibility, which is $0$ for Hamiltonian systems.
For a non-Hamiltonian system the compressibility might be different from $0$.

	\subsection{Phase space metric}
	To study non-Hamiltonian systems the phase-space metric has to be investigated:

	$$\frac{d}{dt}J(x_t;x_0) = J(x_t;x_0)\nabla\cdot\dot{x}_t\Rightarrow\frac{d}{dt}J(x_t;x_0) = J(x_t;x_0)\kappa(x_t, t)$$

	Where $J$ is the Jacobian of the transformation from the coordinates at time $0$ to the ones at time $t$.
	Because of the dependence on compressibility $J$ can be variable.
	Because of this the Liouville theorem $dx_t = dx_0$ no longer holds, but the solution of the derivative above is:

	$$J(x_t;x_0) = e^{\int_0^tds\kappa(x_s, s)}$$

	This is the multiplicative factor that has to be added to the Liouville theorem to obtain the system evolution.
	Considering that there exists a function $w(x_t,t)$ whose derivative is $\kappa(x_t, t)$, so that the integral can be re-written:

	$$\kappa(x_t, t) = \frac{dw(x_t, t)}{dt}\qquad J(x_t; x_0) = e^{w(x_t, t)-w(x_0, 0)}$$

	Applying it to the infinitesimal volume element:

	$$dx_t = J(x_t;x_0)dx_0\Rightarrow dx_t = e^{w(x_t, t) - w(x_0, 0)}dx_0$$

	Multiplying for $e^{-w}$:

	$$e^{-w(x_t, t)}dx_t = e^{-x_0, 0)}dx_0$$

	Obtaining a generalization of Liouville's theorem.
	In this case it is not the volume that is conserved, but it is the weighted phase space volume is conserved, where the factor is $e^{-w(x_t, t)}$.
	This metric allow to define the volume element in space and the phase space is a non Euclidean space or Reimannian manifold, which is locally curved.
	The Jacobian is the ratio of the metric determinant factors:

	$$J(x; y) = \frac{\sqrt{g(y)}}{\sqrt{g(x)}}\Rightarrow J(x_t;x_0) = \frac{\sqrt{g(x_0, 0)}}{\sqrt{g(x_t, t)}}\qquad \sqrt{g(x_t, t)} = e^{-w(x_t, t)}$$

	Any phase space integral or ensemble average should be performed using $\sqrt{g(x)}dx$ as a volume element without time dependence.
	This is because the infinitesimal volume element is always multiplied by the metric factor.
	Because equilibrium solutions are searched there will be no explicit time dependence.

	\subsection{Liouville equation}
	When applying the previous result to Liouville's equation.
	The equation for the time evolution of the probability distribution function should be generalized:

	$$\frac{\partial}{\partial t}(f(x, t)\sqrt{g(x, t)})+\nabla\cdot(\dot{x}\sqrt{g(x, t)}f(x,t)) = 0$$

	Considering the derivative of $J$:

	$$\frac{d}{dt}J(x_t;x_0) = J(x_t;x_0)\kappa(x_t, t)\Rightarrow \frac{d}{dt}\frac{\sqrt{g(x_0, 0)}}{\sqrt{(x_t,t)}} = \frac{\sqrt{g(x_0.0)}}{\sqrt{g(x_t,t)}}\kappa(x_t, t)$$

	From this equation:

	$$\frac{d}{dt}\sqrt{g(x_t, t)} = -\kappa(x_t, t) \sqrt{g(x_t,t)}$$

	Substituting into the generalized Liouville equation:

	$$\frac{\partial}{\partial t} f(x,t) + \dot{x}\cdot\nabla f(x,t) = 0\Rightarrow \frac{d}{dt}f(x,t) = 0$$

	Noticing that partial and total derivatives are different.
	This is the same result of Hamiltonian systems.
	The ensemble distribution function is conserved in a space with a non-trivial metric:

	$$f(x_t, t)\sqrt{g(x_t, t)}dx_t = f(x_0,0)\sqrt{g(x_0,0)}dx_0$$

	From this the total number of conformation can be computed using the microcanonical ensemble.

	\subsection{Equilibrium solutions}
	Considering Liouville's result:

	$$f(x_t, t)\sqrt{g(x_t, t)}dx_t = f(x_0, 0)\sqrt{g(x_0,0)}dx_0$$

	Trying to find an equilibrium solution, this should not present an explicit time dependence:

	$$f(x_t)\sqrt{g(x_t)}dx_t = f(x_0)\sqrt{g(x_0)}dx_0$$

	Averages can be performed at any instant of time with the correct metric factor.
	The same expression as in the equilibrium Liouville equation:

	$$\xi(x)\cdot\nabla f(x) = 0$$

	Since there is no Hamiltonian there are no Poissons brackets.
	However since $\frac{df}{dt} = 0$ a general equilibrium solution can be built inserting all the conservation laws satisfied by the equation of motion.
	Consider $N_C$ conservation laws: $\Lambda_k(x_t)-C_k = 0$, $\frac{d\Lambda_k(x_t)}{dt} = 0$, $k = 1, \dots, N_C$:

	$$f(x) = \prod\limits_{k=1}^{N_C}\delta(\Lambda_k(x_t) - C_k)$$

	The number of microstates available to the systems is determined by $f(x)$ when it is integrated with respect to the conserved volume element $\sqrt{g(x)}dx$.
	Considering the microcanonical partition function:

	$$\mathcal{E} = \int dx\sqrt{g(x)}f(x) = \int dx\sqrt{g(x)}\prod\limits_{k=1}^{N_C}\delta(\Lambda_k(x_t) -C_k)$$

	\subsection{Analysis of non Hamiltonian equations}
	Performing an analysis of the Nos\`e-Hoover equations:

	$$\dot{\vec{r}}_i = \frac{\vec{p}_i}{m_i}\qquad\dot{\vec{p}}_i = \vec{F}_i-\frac{p_\eta}{Q}\vec{p}_i\qquad\dot{\eta} = \frac{p_\eta}{Q}\qquad\dot{p}_\eta = \sum\limits_i\frac{\vec{p}_i^2}{m_i}-dNkT$$

	Given the equations of a non-Hamiltonian system, the kind of simulated ensemble has to be found.
	Using this equations the conserved quantity is:

	$$\mathcal{H}'(\vec{r}, \eta, \vec{p}, p_\eta) = \mathcal{H}(\vec{r}, \vec{p}) + \frac{p_\eta^2}{2Q} + dNkT\eta\qquad \frac{d\mathcal{H}'}{dt} = 0$$

	$\mathcal{H}'(\vec{r}, \eta, \vec{p}, p_\eta)$ is not an Hamiltonian, but a conserved quantity: there is no way to obtain the Nos\`e-Hoover equations applying Hamiltons equations to $\mathcal{H}'(\vec{r}, \eta, \vec{p}, p_\eta)$.
	Computing the compressibility of this equation:

	$$\kappa = \sum\limits_{i=1}^N[\nabla_{\vec{p}_i}\cdot\dot{\vec{p}}_i + \nabla_{\vec{r}_i}\cdot\dot{\vec{r}}_i] + \frac{\partial\dot{\eta}}{\partial\eta} + \frac{\partial\dot{p}_\eta}{\partial p_\eta} = -Nd\frac{p_\eta}{Q} = -Nd\dot{\eta}$$

	Where $d$ is the number of dimensions.
	However, writing the compressibility as a time derivative:

	$$\kappa = \frac{dw}{dt}\Rightarrow w = -dN\eta\Rightarrow\sqrt{g} = e^{-w} = e^{dN\eta}$$

	Which is the metric factor, which has to be applied to any integral performed on space.

	\subsection{Partition function}
	Considering the conserved quantity and the metric factor:

	$$\mathcal{H}'(\vec{r}, \eta, \vec{p}, p_\eta) = \mathcal{H}(\vec{r}, \vec{p}) + \frac{p_\eta^2}{2Q} + dNkT\eta\qquad \sqrt{g} = e^{dN\eta}$$

	The partition function can be written starting from:

	$$\mathcal{E} = \int dx\sqrt{g(x)}\prod\limits_{k=1}^{N_C}\delta(\Lambda_k(x_t) - C_k)$$

	Which translate into:

	$$\mathcal{E}_T(N, V, C_1) = \int d^N\vec{p}\int_{\mathcal{D}(V)}d^n\vec{r}\int dp_\eta d\eta e^{dN\eta}\delta\biggl(\mathcal{H}(\vec{r}, \vec{p}) + \frac{p_\eta^2}{2Q} + dNkT\eta - C_1\biggr)$$

	This function depends on the number of particles, volume and the constant $C_1$.
	And where $\Lambda_k = \mathcal{H}'$ and so temperature is a parameter, $\mathcal{D}(V)$ is the volume of the system.
	Now, considering the $\delta$ function to find a value $\eta_0$ to exploit the property of the $\delta$ function:

	$$\delta\biggl(\mathcal{H}(\vec{r}, \vec{p}) + \frac{p_\eta^2}{2Q} + dNkT\eta - C_1\biggr) = \frac{1}{dNkT}\delta(\eta-\eta_0)\qquad \eta_0 = \frac{1}{dNkT}\biggl(C_1-\mathcal{H}(\vec{r}, \vec{p}) - \frac{p_\eta^2}{2Q}\biggr)$$

	So the $\delta$ function can be re-written:

	$$\mathcal{E}_T(N, V, C_1) = \frac{e^{\beta C_1}}{dNkT}\int d^N\vec{p}\int_{\mathcal{D}(V)}d^N\vec{r}\int dp_\eta e^{-\frac{\beta p_\eta^2}{2Q}}e^{-\beta\mathcal{H}(\vec{r}, \vec{p})}$$

	Now integrating over $p_\eta$ because it is a Gaussian integral:

	$$\mathcal{E}_T(N, V, C_1) = \frac{e^{\beta C_1}\sqrt{2\pi Qk T}}{dNkT}\int d^N\vec{p}\int_{\mathcal{D}(V)}d^N\vec{r}e^{-\beta\mathcal{H}(\vec{r}, \vec{p})}$$

	Finding a pre-factor depending on temperature, but the microcanonical partition function is the canonical partition function, non considering a multiplicative factor.
	Time is rescaled so it is no longer correct and has to be corrected when performing dynamics where time is explicitly stated.
	In the absence of external forces $\mathcal{H}'(\vec{r}, \eta, \vec{p}, p_\eta)$ is not the only conserved quantity.
	There are three conservation laws, because the momentum is conserved in each direction:

	$$\vec{P} = \sum\limits_{i=1}^N\vec{p}_i\qquad \vec{K} = \vec{P}e^\eta\Rightarrow \frac{d\vec{K}}{d t} = 0$$

	The extra factor $e^\eta$ and so $\vec{K}$ is conserved.
	And these should be included as $\delta$ factors.

		\subsubsection{Harmonic oscillator}
		Performing a simulation of the harmonic oscillator in one dimension then, applying the Nos\`e-Hoover equations a strange result is obtained.
		This is completely wrong because where there is a maximum there is $0$.
		The reason for this is the fact that $\mathcal{H}'$ is not the only conserved quantity.

	\subsection{Nos\`e-Hoover chains}
	The trick to solve the problem a huge number of $\eta$ variables all the possible conservation laws should be included.
	This creates Nos\`e-Hoover chains: extra variables that are thremostatting each extra-variable, so $\eta$ become $\eta_1$ and $\eta_2$ takes care of $\eta_1$ and so on up to $M$.

	\begin{multicols}{2}
		\begin{itemize}
			\item $\dot{\vec{r}}_i = \frac{\vec{p}_i}{m_i}$.
			\item $\dot{\vec{p}}_i = \vec{F}_i - \frac{p_{\eta_1}}{Q_1}\vec{p_i}$.
			\item $\dot{\eta}_j = \frac{p_{\eta_j}}{Q_j}\quad j = 1, \dots, M$.
			\item $\dot{p}_{\eta_1} = \biggl[\sum\limits_i\frac{\vec{p}_i^2}{m_i}-dNkT\biggr] - \frac{p_{\eta_2}}{Q_2}p_{\eta_1}$.
			\item $\dot{p}_{\eta_j} = \biggl[\sum\limits_i\frac{\vec{p}_{\eta_{j-1}}^2}{Q_{j-1}}-kT\biggr] - \frac{p_{\eta_{j+1}}}{Q_{j+1}}p_{\eta_j}\quad j = 2, \dots, M-1$.
			\item $\dot{p}_{\eta_M} = \frac{p^2_{\eta_{M-1}}}{Q_{M-1}}-kT$.
		\end{itemize}
	\end{multicols}

	This is done so each extra variable is exactly weighted with the kinetic energy, all the parameters $Q_i$ has to be found.
	The optimal choice for the $Q$s: $Q_1 = dNkT\tau^2$ and $Q_j = kT\tau^2$, with $\tau$ the characteristic time.
	With this plausible equations the partition function can be re-written.
	Now from the compressibility of these equation the kind of ensemble can be found.
	The conserved quantity becomes:

	$$\mathcal{H}'(\vec{r}, \eta, \vec{p}, p_\eta) = \mathcal{H}(\vec{r}, \vec{p}) + \sum\limits_{j=1}^M\frac{p_{\eta_j}^2}{2Q_j} + dNkT\eta_1 + kT\sum\limits_{j=2}^M\eta_j$$

	Computing the compressibility:

	$$\kappa = -dN\frac{P_{\eta_1}}{Q_1} - \sum\limits_{j=2}^M\frac{p_{\eta_j}}{Q_j} = -dN\dot{\eta}_1 - \dot{\eta}_c\qquad \eta_c = \sum\limits_{j=2}^M\eta_j$$

	Then, to obtain the metric factor:

	$$\kappa = \frac{dw}{dt}\Rightarrow w = -dN\eta_1 - \eta_c\Rightarrow \sqrt{g} = e^{-w} = e^{dN\eta_1 + \eta_c}$$

		\subsubsection{Partition function}
		Now the partition function can be computed.
		Considering the conserved quantity:

		$$\mathcal{H}'(\vec{r}, \eta, \vec{p}, p_\eta) = \mathcal{H}(\vec{r}, \vec{p}) + \sum\limits_{j=1}^M\frac{p_{\eta_j}^2}{2Q_j} + dNkT\eta_1 + kT\sum\limits_{j=2}^M\eta_j$$

		And the metric factor:

		$$\sqrt{g} = e^{dN\eta_1 + \eta_c}$$

		$$\mathcal{E}_T(N, V, C_1) = \int d^N\vec{p}\int_{\mathcal{D}(V)}d^N\vec{r}\int d^Mp_\eta d^M\eta e^{dN\eta_1 + \eta_c}\delta \biggl(\mathcal{H}(\vec{r}, \vec{p}) + \sum\limits_{j=1}^M\frac{p_{\eta_j}^2}{2Q_j} + dNkT\eta_1 + kT\eta_c - C_1\biggr)$$

		$$\delta\biggl(\mathcal{H}(\vec{r}, \vec{p}) + \sum\limits_{j=1}^M\frac{p_{\eta_j}^2}{2Q_j} + dNkT\eta_1 + kT\eta_c - C_1\biggr) = \frac{1}{dNkT}\delta(\eta_1-\eta_0)\qquad \eta_0 = \frac{1}{dNkT}\biggl(C_1-\mathcal{H}(\vec{r},\vec{p})-\sum\limits_{j=1}^M\frac{p_{\eta_j}^2}{2Q_j} - kT\eta_c\biggr)$$

		$$\mathcal{E}_T(N, V, C_1) = \frac{e^{\beta C_1}}{dNkT}\int d^N\vec{p}\int_{\mathcal{D}(V)}d^N\vec{r}\prod\limits_{j=1}^M\biggl(\int dp_\eta e^{-\beta\frac{p_{\eta J}^2}{2Q_j}}\biggr)\prod\limits_{j=2}^M\biggl(\int d\eta_j\biggr)e^{-\beta\mathcal{H}(\vec{r},\vec{p})}$$

		Assuming that all the $\eta_j$ are limited:

		$$\mathcal{E}_T(N, V, C_1) = \mathcal{M}\int d^N\vec{p}\int_{\mathcal{D}(V)}d^N\vec{r}e^{-\beta\mathcal{H}(\vec{r}, \vec{p})}$$

		In this way the canonical distribution is obtained.
		Using the Hoover-Chains simulate exactly the canonical ensemble.

		\subsubsection{Harmonic oscillator}
		It can be seen now hot the Nos\`e-Hoover chains allow to simulate the harmonic oscillator exactly.
		Going on on the Nos\`e-Hoover chains points in the canonical ensemble is being simulated, but not in the exact sequence of time.
		This doesn't allow to compute dynamic properties of the system.
