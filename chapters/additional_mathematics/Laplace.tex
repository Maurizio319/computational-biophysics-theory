\section{Laplace transform} \label{chap: laplace}

A very detailed explanation is written in \href{https://www.youtube.com/watch?v=n2y7n6jw5d0&pp=ygUcYnJpYWxsaWFudCBsYXBsYWNlIHRyYW5zZm9ybQ%3D%3D}{link}.

It can be used to convert differential equations ino algebraic equations. It is written as

\begin{equation}
    L\{f(t)\} = F(s)
\end{equation}

with

\begin{equation}
    F(s) = \int_0^\infty{e^{-st} f(t) dt}
\end{equation}

Therefore, it produces a new function out of the first one given in input.\\

It has the following properties:
\begin{itemize}

    \item \textbf{Linearity}: $L\{a f(t) + b g(t)\} = a L\{f(t)\} + b L\{g(t)\}$ 
    \item \textbf{Existence}: Given that f(t) is of "exponential order" if $|f(t)| \le M e^{ct}$ for large t, some M and c. Therefore $\lim_{t \rightarrow \infty}{\frac{f(t)}{e^{ct}}}$. The existence theorem says that if f(t) is continous and exponential order     with constant c, then
    $$
        F(s) = L\{f(t)\}
    $$
    is defined for all $s > c$. In practice, the F(s) value converges
    \item \textbf{Inverses}: It is possible to obtain $f(t)$ such that $F(s) = L\{f(t)\}$. ANd it exists a unique answer.
    Therefore, it is possible to perform

    \[
    f(t) = L^{-1}\{F(s)\}    
    \]

    \item \textbf{Translation property}: If \( \mathcal{L}\{f(t)\} = F(s) \), then \( \mathcal{L}\{e^{at} f(t)\} = F(s - a) \). Indeed,

    \begin{align*}
        \mathcal{L}\{e^{at} f(t)\} &= \int_0^\infty e^{-st} \cdot e^{at} f(t) \, dt \\
        &= \int_0^\infty e^{-(s - a)t} f(t) \, dt \\
        &= F(s - a)
    \end{align*}
\end{itemize}


\textbf{e.g.}
\begin{align*}
    L\{t^n\} &= F(s) = \int_0^\infty{e^{-st} t^n dt} \\
    &= \int_0^\infty{e^{-u} \left(\frac{u}{s}\right)^n \frac{du}{s}}\\
    &= \left(\int_0^\infty{e^{-u} u^n du}\right) \frac{1}{s^{n+1}}\\
    &= \frac{1}{s^{n+1}} \Gamma(n+1) = \frac{n!}{s^{n+1}}
\end{align*}

\textbf{e.g.:} Given the partition function $Q_N(V, T) = \int{g(\epsilon) e^{-\beta \epsilon} d\epsilon}$, You notice that $Q_N(V, T) = \int{g(\epsilon) e^{-\beta \epsilon} d\epsilon} = L\{f(\epsilon)\}$. Therefore

$$
L\{f(\epsilon)\} = \frac{8 \pi V}{h^3} \frac{1}{\beta^3 c^3} = \int{g(\epsilon) e^{-\beta \epsilon} d\epsilon}
$$

You can observe that the value of $s$ is $\beta$, therefore

$$
F(s) \propto \frac{1}{\beta^3}
$$

This form is similar to the known form

$$
F(s) = \frac{n!}{s^{n+1}} \;\; s>0
$$

which is obtained when $f(t) = t^n$.

\begin{align*}
    F(\beta) &\propto \frac{1}{\beta^3} \\
    &= \frac{2!}{\beta^{2+1}} \cdot \frac{1}{2} \\
    &= \frac{1}{2} \cdot \frac{2}{\beta^{2+1}}
\end{align*}

Therefore, n=2 and $g(\epsilon)$ ($f(t)$) is

$$
g(\epsilon) = \left(\frac{4\pi V}{h^3 \beta^3 c^3}\right) \epsilon^2
$$

\subsubsection{Laplace transform of derivatives}

SUppose that $f(t)$ is continous, piece-wise smooth, an of exponential order.


\begin{align*}
    L\{f'(t)\} &= \int_0^\infty{e^{-st}f'(t)dt} \\
    &= \left(e^{-st} f(t)\right)_0^\infty - \int_0^\infty{(-s) e^{-st} f(t) dt} \\
    &=  - f(0) + s F(s) = -f(0) + sL\{f(t)\}
\end{align*}

Similarly

\begin{align*}
    L\{f''(t)\} &= L\{g'(t)\}\\
    &= -g(0)  + s L\{g(t)\} \\
    &= - f'(0) + s L\{f'(t)\} \\
    &= -f'(0) + s\left[-f(0) + s L\{f(t)\} \right]\\
    &= s^2 L\{f(t)\} - sf(0) - f'(0)
\end{align*}

